\documentclass[mathserif]{beamer}
\usepackage{beamerthemesplit}
\usepackage{amsmath}
\usepackage{amsfonts}
\begin{document}
\begin{frame}[shrink=20]
Requiring the partition function to be modular invariant, gives some properties to $\epsilon(\alpha,\beta)$.
\begin{itemize}
\item Under $\tau \rightarrow \tau+1$ 
$$
\{\alpha,\beta\} \rightarrow e^{-i\pi n(\alpha)/8}\{\alpha,\bar \alpha \beta\}.
$$
This will require $\epsilon(\alpha,\beta)=\varepsilon_\alpha\epsilon(\alpha,\bar \alpha \beta)$.
\item Under $\tau \rightarrow -\frac{1}{\tau}$ 
$$
\{\alpha,\beta\} \rightarrow e^{i\pi n(\alpha \cap \beta)/4}\{\beta, \alpha\}.
$$
This will require $\epsilon(\alpha,\beta)=\delta_\alpha\delta_\beta\epsilon(\beta, \alpha)\varepsilon_{\alpha \cap \beta}^{-2}$.
\end{itemize}
In the equations above $\varepsilon_\alpha=e^{-i\pi n(\alpha)/8}$.
\end{frame}
\end{document}
