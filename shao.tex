\let\nofiles\relav
\documentclass{article}
%\documentclass{ctexart}
\usepackage{amsmath}
\usepackage{amssymb}
\usepackage{amsthm}
\usepackage{graphicx}
\usepackage{enumerate}
\usepackage{enumitem}
%\usepackage{multirow}
\usepackage{tikz}
\def\degree{{}^{\circ}}
\usepackage[numbers]{natbib}
\begin{document}
\title{Imaging the Ionosphere by Assimilating Observations From Multi Sources/Platform and Error Sources Analysis}
\author{Naifeng Fu |SHAO\vspace{-2cm}}
% \thanks{nffu@shao.ac.cn}\\
% \and Peng Guo\\SHAO\vspace{-2cm}\\
% \and Menjie Wu\\SHAO\vspace{-2cm}\\
% \and Xiaogong Hu\\SHAO\vspace{-2cm}\\
% }

\maketitle
%\tableofcontents

\begin{abstract}
%\ref{abstract}
\par In this paper, using the IRI~\cite{Bilitza07} model as the real field, the ionosphere observation data of the GNSS-IGS station on May 6, 2016 was simulated. With the Nequick~\cite{Nava08} model as the background field, the $ 2.5 \degree \times 5 \degree \times 13 layer \times 1h$ of the global ionospheric density field was constructed by the KF filter algorithm, and the following work was processed : 1. Analyzed various errors and influences in the ionosphere inversion, especially the easily overlooked errors, and proposed and verified the corresponding improvement methods; 2 The effects of multi-system data and multi-source data observations on the observation quality and spatial distribution configuration in ionospheric inversion are analyzed.
\newline
\textbf{Keywords:} data assimilation,GNSS ,COSMIC,electron density,radio occultation.
\end{abstract}
\section{Introduction}
% exist problem
\par In 2013,Ludger ,etc. telled that ...~\cite{Galindo13}, telling that the opinion~\cite{Ludger09} of Ludger developed at 2009 is incorrect.
\section{Data}

%
\subsection{Satellite constellation and Stations}

\subsection{Ionospheric empirical models}

\subsection{Observing systems}

\section{Method}

\subsection{the Reconstruction Model}
%
\par The main Kalman filter equation that we described is as follows:
\begin{equation}\label{e:Kalman filter}
 X_a=X_b+P_{ne} H^t[\lambda^2 H P_{ne} H^t +R_{obs}]^{-1}(Y-HX_b)
\end{equation}
where $X_b$ and $X_a$ are the prior and assiminated electron densities,respectively. $P_{ne}$ and $R_{obs}$ are the error covariances
of the background model and observations, respectively. $H$ is the observation matrix and and $Y$ is the observation vector.
\subsection{the Error Sources and the Error Classification}
\par Kalman filter Function[~\ref{e:Kalman filter}] with the order of the observation number.
\subsubsection{The Influence of the Inversion Algorithm's Error}
% \lambda -> setting
% obscov/modelcov
\subsubsection{The Influence of the Model Assumption's Error}
\par three factors is showed as follow:\\
\midskip
\begin(itemize)
\item[-] The influence and correction of The top ionosphere and plasma layer\\
\item[-] The influence and correction of the ionospheric time variations\\
\item[-] The influence and correction of the ionospheric grid representation\\
\end(itemize)
\subsection{The Role of the Multi-constellation and Multi-observation Systems}
\section{Result}

\subsection{The Improvement of the Multi-constellation and Multi-observation Systems}

\subsection{The Improvement of the Corrections Above}

\section{Conclusion}



 \bibliographystyle{plain}
 \bibliography{paperref}

\end{document}
